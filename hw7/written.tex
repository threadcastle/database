\documentclass[10pt]{article}
\usepackage{fullpage}
\usepackage{amsmath,amsfonts,amsthm}
\usepackage{graphicx}
\usepackage{listings}
\usepackage{float}
\usepackage{listings}

\graphicspath{ {./figure/} }

\usepackage{enumitem}
\usepackage{amssymb}
\usepackage{amsmath}
\usepackage{fancyvrb}
\usepackage{venndiagram}
\usepackage{tikz}
\newtheorem{example}{Example}
\usepackage{alltt,xcolor}


\newcommand{\redbullet}{$\textcolor{red}{\bullet}$}
\newcommand{\bluebullet}{$\textcolor{blue}{\bullet}$}

\newcommand{\red}[1]{\textcolor{red}#1}
\newcommand{\green}[1]{{\color{green}#1}}
\newcommand{\blue}[1]{{\color{blue}#1}}
\newcommand{\purple}[1]{{\color{purple}#1}}
\newcommand{\orange}[1]{{\color{orange}#1}}
\newcommand{\hashjoin}[1]{\begin{array}{c} \mathtt{hashJoin} \\ \bowtie\\ {#1}\\ \end{array}}
\newcommand{\mergejoin}[1]{\begin{array}{c} \mathtt{mergeJoin} \\ \bowtie\\ {#1}\\ \end{array}}
\newcommand{\indexjoin}[1]{\begin{array}{c} \mathtt{indexJoin} \\ \bowtie\\ {#1}\\ \end{array}}
\newcommand{\nestedjoin}[1]{\begin{array}{c} \mathtt{nestedLoopJoin} \\ \bowtie\\ {#1}\\ \end{array}}

\newcommand{\sortprojection}[1]{\begin{array}{c} \mathtt{sortProjection} \\ \pi_{#1}\ \end{array}}
\newcommand{\hashprojection}[1]{\begin{array}{c} \mathtt{hashProjection} \\ \pi_{#1}\ \end{array}}
\newcommand{\scan}[1]{\begin{array}{c} \mathtt{scan} \\ {#1}\ \end{array}}
\newcommand{\filter}[1]{\begin{array}{c} \mathtt{filterSelection}\\ \sigma_{#1}\ \end{array}}
\newcommand{\filterscan}[2]{\begin{array}{c} \mathtt{filterScan}\\ \sigma_{#1}(#2)\ \end{array}}
\newcommand{\sortfilterscan}[2]{\begin{array}{c}\mathtt{sort} \mathtt{filterScan}\\ \sigma_{#1}(#2)\ \end{array}}



\author{Team 37}
\title{Homework 7 -- Solutions}

\begin{document}

\maketitle
\subsection*{1.}
\subsubsection*{(a)}
Towards query plan for the NO generalized quantifier (Pure SQL
with true = not some)

\begin{alltt}\textcolor{blue}{
explain select r1.a,r2.a
        from r r1,r r2
        where not true=some(select r3.a=r1.a and true=some(select r4.a=r2.a and r3.b=r4.b
		                                                        from r r4)
			                       from r r3);}
\end{alltt}

{\footnotesize
\begin{alltt}
\textcolor{purple}{
                              QUERY PLAN
------------------------------------------------------------------------
 Nested Loop  (cost=0.00..143103268435.85 rows=2553800 width=8)
   Join Filter: (NOT (SubPlan 2))
   ->  Seq Scan on r r1  (cost=0.00..32.60 rows=2260 width=4)
   ->  Materialize  (cost=0.00..43.90 rows=2260 width=4)
         ->  Seq Scan on r r2  (cost=0.00..32.60 rows=2260 width=4)
   SubPlan 2
     ->  Seq Scan on r r3  (cost=0.00..56029.75 rows=2260 width=1)
           SubPlan 1
             ->  Seq Scan on r r4  (cost=0.00..43.90 rows=2260 width=1)
}
\end{alltt}
}

Four nested loops: $O(|R|^4)$
\subsubsection*{(b)}
Towards query plan for the NO generalized quantifier (Pure SQL
with not exists)

\begin{alltt}\textcolor{blue}{
explain select r1.a,r2.a
        from r r1,r r2
        where not exists(select 1
    				from r r3
    				where r3.a=r1.a and exists(select 1
    										  from r r4
    										  where r4.a=r2.a and r3.b=r4.b));
}
\end{alltt}

{\footnotesize
\begin{alltt}
\textcolor{purple}{
                                QUERY PLAN
--------------------------------------------------------------------------
 Hash Anti Join  (cost=60.85..168313587.91 rows=3830700 width=8)
   Hash Cond: (r1.a = r3.a)
   Join Filter: (hashed SubPlan 2)
   ->  Nested Loop  (cost=0.00..63915.85 rows=5107600 width=8)
         ->  Seq Scan on r r1  (cost=0.00..32.60 rows=2260 width=4)
         ->  Materialize  (cost=0.00..43.90 rows=2260 width=4)
               ->  Seq Scan on r r2  (cost=0.00..32.60 rows=2260 width=4)
   ->  Hash  (cost=32.60..32.60 rows=2260 width=8)
         ->  Seq Scan on r r3  (cost=0.00..32.60 rows=2260 width=8)
   SubPlan 2
     ->  Seq Scan on r r4  (cost=0.00..32.60 rows=2260 width=8)
}
\end{alltt}
}
$O(2R^2+R)$
\subsubsection*{(c)}
Query plan for NO quantifier using IN

\begin{alltt}\textcolor{blue}{
explain select r1.a,r2.a
        from r r1,r r2
        where not exists(select 1
				        from r r3
				        where r3.a=r1.a and (r2.a,r3.b) in (select r4.a,r4.b
												   from r r4));
}
\end{alltt}

{\footnotesize
\begin{alltt}
\textcolor{purple}{
                                QUERY PLAN
--------------------------------------------------------------------------
 Hash Anti Join  (cost=99.10..165049.66 rows=3830700 width=8)
   Hash Cond: (r1.a = r3.a)
   Join Filter: (hashed SubPlan 1)
   ->  Nested Loop  (cost=0.00..63915.85 rows=5107600 width=8)
         ->  Seq Scan on r r1  (cost=0.00..32.60 rows=2260 width=4)
         ->  Materialize  (cost=0.00..43.90 rows=2260 width=4)
               ->  Seq Scan on r r2  (cost=0.00..32.60 rows=2260 width=4)
   ->  Hash  (cost=32.60..32.60 rows=2260 width=8)
         ->  Seq Scan on r r3  (cost=0.00..32.60 rows=2260 width=8)
   SubPlan 1
     ->  Seq Scan on r r4  (cost=0.00..32.60 rows=2260 width=8)
}
\end{alltt}
}

$O(2R^2+R)$

\subsection*{2.}

\begin{itemize}
\item[a.] use $true = all()$ and $exists$\\



\begin{lstlisting}
"qp"
"Unique"
"  ->  Sort"
"        Sort Key: p.a, q.a"
"        ->  Hash Semi Join"
"              Hash Cond: (p.a = r.a)"
"              Join Filter: (SubPlan 1)"
"              ->  Nested Loop"
"                    ->  Seq Scan on p"
"                    ->  Materialize"
"                          ->  Seq Scan on q"
"              ->  Hash"
"                    ->  Seq Scan on r"
"              SubPlan 1"
"                ->  Seq Scan on s"
"                      Filter: (a = q.a)"
\end{lstlisting}


\newpage
\item[b.] eliminate $true = all()$ and $ [not] exists $ and using RA \\
\begin{lstlisting}
"qp"
"Unique"
"  ->  Subquery Scan on m"
"        ->  SetOp Except"
"              ->  Sort"
"                    Sort Key: ""*SELECT* 1"".pa, ""*SELECT* 1"".qa, 
                        ""*SELECT* 1"".a, ""*SELECT* 1"".b"
"                    ->  Append"
"                          ->  Subquery Scan on ""*SELECT* 1"""
"                                ->  Nested Loop"
"                                      ->  Hash Join"
"                                            Hash Cond: (p.a = r.a)"
"                                            ->  Seq Scan on p"
"                                            ->  Hash"
"                                                  ->  Seq Scan on r"
"                                      ->  Materialize"
"                                            ->  Seq Scan on q"
"                          ->  Subquery Scan on ""*SELECT* 2"""
"                                ->  Hash Join"
"                                      Hash Cond: (s.b = r_1.b)"
"                                      ->  Hash Join"
"                                            Hash Cond: (q_1.a = s.a)"
"                                            ->  Seq Scan on q q_1"
"                                            ->  Hash"
"                                                  ->  Seq Scan on s"
"                                      ->  Hash"
"                                            ->  Hash Join"
"                                                  Hash Cond: (p_1.a = r_1.a)"
"                                                  ->  Seq Scan on p p_1"
"                                                  ->  Hash"
"                                                        ->  Seq Scan on r r_1"
\end{lstlisting}

\newpage
\item[c.] eliminate $where$ and using $join$ $ operation$\\
\begin{lstlisting}
"qp"
"Unique"
"  ->  Subquery Scan on m"
"        ->  SetOp Except"
"              ->  Sort"
"                    Sort Key: ""*SELECT* 1"".pa, ""*SELECT* 1"".qa, 
                        ""*SELECT* 1"".a, ""*SELECT* 1"".b"
"                    ->  Append"
"                          ->  Subquery Scan on ""*SELECT* 1"""
"                                ->  Nested Loop"
"                                      ->  Hash Join"
"                                            Hash Cond: (p.a = r.a)"
"                                            ->  Seq Scan on p"
"                                            ->  Hash"
"                                                  ->  Seq Scan on r"
"                                      ->  Materialize"
"                                            ->  Seq Scan on q"
"                          ->  Subquery Scan on ""*SELECT* 2"""
"                                ->  Hash Join"
"                                      Hash Cond: (s.b = r_1.b)"
"                                      ->  Hash Join"
"                                            Hash Cond: (s.a = q_1.a)"
"                                            ->  Seq Scan on s"
"                                            ->  Hash"
"                                                  ->  Seq Scan on q q_1"
"                                      ->  Hash"
"                                            ->  Hash Join"
"                                                  Hash Cond: (p_1.a = r_1.a)"
"                                                  ->  Seq Scan on p p_1"
"                                                  ->  Hash"
"                                                        ->  Seq Scan on r r_1"
\end{lstlisting}

\end{itemize}


\newpage
\subsection*{3.}
\subsection*{3.}
\subsubsection*{(a)}
We compare the non-optimized query Q3
with the optimized query Q4 for various differently-sized relations R. In the
following table are some of these comparisons for various differently-sized random relations R

\begin{center}
\title{results on table.}
\\ \hspace*{\fill} \\
\begin{tabular}{c|c|c}
\hline
dataset scale & execution time of Q3 (in ms) & execution time of Q4 (in ms) \\ \hline
$10^4$ rows &  35.493 & 27.404 \\
$10^5$ rows & 623.002 & 967.326 \\
$10^6$ rows & 8144.356 & 13829.844 \\
\hline
\end{tabular}
\end{center}
\subsubsection*{(b)}
Sometimes  the effectiveness of query optimization will not be better than the effectiveness of non-optimized query.
\subsection*{4-6}
In question 4-6, we consider three scenarios.
\begin{itemize}
\item[a.] Size of relation $P$ changes\\
Size for $P$ ranges from 6,000 to 600,000. $Q$ is generated by function SetOfIntegers with 10,000 as the upper-bound, including more than 6,000 tuples. $R$ is generated with more than 9,000 tuples.

\item[b.] Size of relation $R$ changes and is always greater than size of $P,Q$.\\
Size for $R$ ranges from 10,000 to 100,000. $P, Q$ are generated by function SetOfIntegers with 10,000 as the upper-bound, including more than 6,000 tuples.


\item[c.] Size of relation $R$ changes and the $a$ attribute in R is a foreign key referencing the $a$ attribute in P\\
Size for $R$ ranges from 10,000 to 100,000. $P, Q$ are generated by function SetOfIntegers with 10,000 as the upper-bound, including more than 6,000 tuples. Besides, $a$ attribute in R is a foreign key referencing the $a$ attribute in P.

\end{itemize}

\subsection*{CASE A}
\begin{center}
\begin{tabular}{c|r|r|r}
size $n$ of relation $P/$ \\
execution time of $Q$ (in ms)
&
6,000&
60,000&
600,000
 \\ \hline
$Q_5$    & 4.53 &  15.3 & 99.4\\
$Q_6$    & 15.5 &  86.6 & 699.2\\
$Q_{5-2}$& 6551  & 62710 & 606682\\
$Q_7$    & 4.03 &  20.1 & 2321\\
$Q_8$    & 14.3  & 94.5 & 1045\\
\end{tabular}
\end{center}

\subsection*{CASE B}
\begin{center}
\begin{tabular}{c|r|r|r}
size $n$ of relation $R/$ \\
execution time of $Q$ (in ms)
&
10,000&
100,000&
1,000,000
 \\ \hline
$Q_5$    & 4.977 &  23.8 & 131951\\
$Q_6$    &15.916 &  47.9 & 279.6\\
$Q_{5-2}$& 6990 &  60434 & 408192\\
$Q_7$    & 5.15 &  23.5 & 474798\\
$Q_8$    & 15.04 &  96.7 & 199\\
\end{tabular}
\end{center}

\subsection*{CASE C}
\begin{center}
\begin{tabular}{c|r|r}
size $n$ of relation $R/$ \\
execution time of $Q$ (in ms)
&
10,000&
100,000
 \\ \hline
$Q_5$    & 5.07 &  27.5 \\
$Q_6$    & 25.8 &  52.3 \\
$Q_{5-2}$& 8446 &  8903 \\
$Q_7$    & 3.9 &   24.4 \\
$Q_8$    & 15.7 &  34.0 \\
\end{tabular}
\end{center}

<<<<<<< HEAD
\subsection*{4. Comparison between $Q_5,  Q_6$}
$Q_5$ evaluate the existence in a nested query. As the size of $P$ increases, the execution time grows slowly while the time grows sharply with increased size of $R$. In the first case, the relation $P$ is partial evaluated with a small $R$. In the second case, the time grows due to repeatedly scanning of the large relation $R$. When a foreign key constraint is added between $P, R$, the performance for $Q_5$ doesn't improve a lot.\\
$Q_6$ is a RA translated from $Q_5$. The equality of attributes is optimized with \textit{natural join} among relations. It has to be scan all tuples to get a joined subset satisfying the condition, therefore it has a worse performance compared to $Q_5$. When the foreign key constraint is considered, the performance of $Q_6$ improves dramatically ($Q_6$ in case b versus case c).\\
In general, $Q_5$ has greater execution time than $Q_6$ in most cases. However, the magnitude of running time are quite similar with the same relations in both queries.
\subsection*{5. Comparison between $Q_{5-2},  Q_6$}
$Q_{5-2}$ is a pure sql consisting \textit{true = all, true= some} statement. It is less efficient than $Q_5$ and the execution time linearly increases as the size of relation goes up. The optimized RA of $Q_{5-2}$ is identical as $Q_6$, which is the optimized RA of $Q_5$. $Q_6$ shows huge improvement of performance compared to $Q_{5-2}$ in all cases.\\
$Q_{5-2}$ fully evaluates within a nested query. Therefore, it will have a poor performance compared to set difference process in $Q_6$.
\subsection*{6. Comparison between $Q_7,  Q_8$}
$Q_7$ is a pure sql that allows partially evaluation within the sub-query. In comparison, $Q_8$ is a RA involved multiple set differentiation. In most case, $Q_7$ outperforms $Q_8$. However, We observed huge improvement of $Q_8$ when there is a large relation $P$, or the size of $R$ is much greater than size of $P, Q$. The foreign key constraint is beneficial for $Q_8$ but there are few difference for $Q_7$.
\subsection*{7. Comparison between Problem 4 and Problem 6}
$Q_5$ has a similar performance as $Q_7$. In both queries, the running time decreases due to violations of subquery, leading to a early termination without fully scanning.\\
$Q_6$ and $Q_8$ are RA with multiple set difference, but slightly different. $Q_6$ involves several join operations and nested selections to remove the attribute equity among relations. In contrast, $Q_8$ has only one \textit{cross join} operation and less subquery scanning. Therefore, $Q_8$ has a better performance, especially with large relations. 
=======
\subsection*{7.}

>>>>>>> 5b4fd3d08784a81de3853e981d8469222d1acd45
\newpage
\subsection*{Appendix.} 

\subsection*{2.}


using $true = all()$ and $ [not] exists $ 
{
\begin{alltt}\textcolor{blue}{

select distinct p.a, q.a from P p, Q q where exists(
	select 1 from R r where r.a = p.a and true = all(select s.b <> r.b from S s where s.a = q.a));
}
\end{alltt}
}



eliminate $true = all()$ and using RA
{
\begin{alltt}\textcolor{blue}{
select distinct m.pa, m.qa from
    (select p.a as pa , q.a as qa, r.* from P p, Q q, R r
    where r.a = p.a
    
except 

select p.a as pa, q.a as qa, r.* from P p, Q q, R r, S s
    where r.a = p.a and s.a = q.a and s.b = r.b) m;}
\end{alltt}
}



eliminate $where$ and using $join$ $ operation$
{
\begin{alltt}\textcolor{blue}{

select distinct m.pa, m.qa from
(select p.a as pa , q.a as qa, r.* from (P p natural join R r) cross join Q q
except 
select p.a as pa, q.a as qa, r.* from (R r natural join P p) join (S s natural join Q q) on (s.b = r.b)) m;
}
\end{alltt}
}
\subsection*{4.}
Consider the query $Q_5$ 
{
\begin{alltt}\textcolor{blue}{
select p.a
from   P p
where  p.a not in (select r.a
                   from   R r
                   where  r.b not in (select q.b
                                      from   Q q));}
\end{alltt}
}
$Q_5$ is translated to RA and optimized.
{
\begin{alltt}\textcolor{blue}{
select p
from P p
except
(
select distinct p
from P p natural join (select distinct r.a as a from R r)r
except
select distinct p
from P p natural join (select distinct r.a as a 
                       from (R r natural join (select distinct q.b as b from Q q)q)r
                       )r;}
\end{alltt}
}
\subsection*{5.}
Rewrite $Q_5$
{
\begin{alltt}\textcolor{blue}{
select p.a
from P p
where true = all (select p.a != r.a or true = some (select r.b = q.b
from Q q)
from R r);}
\end{alltt}
}
The translated and optimized query is identical as  $Q_6$ in question 4.
\subsection*{6.}
$Q_7$ is translated to RA and optimized.
{
\begin{alltt}\textcolor{blue}{
select p.a
from P p
where not exists (select 1
from Q q
where (p.a, q.b) not in (select r.a, r.b
from R r));
}
\end{alltt}
}

Rewrite $Q_7$ as optimized RA
{
\begin{alltt}\textcolor{blue}{
select p.a
from P p
except
select q1.a
from
(
select p.a, q.b from P p cross join Q q
except
select * from R r
)q1;
}
\end{alltt}
}
\end{document}
